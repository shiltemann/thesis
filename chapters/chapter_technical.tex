\begin{savequote}[75mm]
Nulla facilisi. In vel sem. Morbi id urna in diam dignissim feugiat. Proin molestie tortor eu velit. Aliquam erat volutpat. Nullam ultrices, diam tempus vulputate egestas, eros pede varius leo.
\qauthor{Quoteauthor Lastname}
\end{savequote}

\chapter{Bioinformatics Tools}
\label{chapter:general}
\setcounter{figure}{-1}
\setcounter{table}{-1}
\setcounter{section}{-1}
%\setcounter{enumiv}{-1}
\setcounter{NAT@ctr}{-1}

\begin{comment}
iReport, Circos?, myFAIR?, Galaxy2018update? (I am co-author and technically everybody is
shared first on that? would fit nicely with the overall story as every other paper connects
to Galaxy.)
- Training paper in its own section or add it here?
\end{comment}

Bioinformatics plays a role in a wide range of research fields. While the biology might be vastly different, many of the underlying bioinformatics concepts are shared and may be reused regardless of the scientific application. This chapter covers a number of such generic bioinformatics applications.

The vast majority of bioinformatics tools are unux-based commandline applications. This offers great power and flexibility, but with this power also comes great complexity and often a bioinformatician is needed to run the analyses, rather than the domain expert who can best interpret results. The Galaxy project aims to improve the user-friendliness of such tools by providing a graphical user interface to commandline unix tools.

Analysis pipelines often generate a large number of outputs, and with the explosion of data in next generation sequencing field, many of these resulting outputs contain millions of lines and can no longer easily be reviewed manually. Visualisation and summarisation of results has become an integral part of any analysis pipeline. The iReport tool provides generic reporting within the Galaxy framework. Circos offers visualisation for high-dementional data.

\begin{comment}
Contents of this chapter:
\begin{enumerate}
\itemsep-0.5em
\item The Galaxy platform for accessible, reprodu- cible and collaborative biomedical analyses: 2018 update
\item iReport: A generalised Galaxy solution for integrated experimental reporting
\item Galactic Circos: Visualisation of high-dimensional data in Galaxy
\end{enumerate}
\end{comment}

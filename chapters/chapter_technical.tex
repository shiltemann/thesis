\begin{savequote}[75mm]
"I was taught that the way of progress was neither swift nor easy."
\qauthor{Marie Curie}
\end{savequote}

\chapter{Accessible Bioinformatics}\label{chapter:general}
\setcounter{figure}{-1}
\setcounter{table}{-1}
\setcounter{section}{-1}
%\setcounter{enumiv}{-1}
\setcounter{NAT@ctr}{-1}

\begin{comment}
iReport, Circos?, myFAIR?, Galaxy2018update? (I am co-author and technically everybody is
shared first on that? would fit nicely with the overall story as every other paper connects
to Galaxy.)
- Training paper in its own section or add it here?
\end{comment}

The vast majority of bioinformatics tools are UNIX-based commandline applications, that are not easy to use for non-bioinformaticians. This chapter describes two tools developed to increase the accessibility of bioinformatics pipelines for non-experts, in the following sub-chapters:

\begin{enumerate}
\itemsep-0.5em
\item \textbf{The Galaxy platform} for accessible, reproducible and collaborative biomedical analyses: 2018 update
\item \textbf{iReport}: A generalised Galaxy solution for integrated experimental reporting
\item \textbf{Galactic Circos} \verb+[maybe. we are finalizing manuscript now]+
\end{enumerate}

The Galaxy project aims to improve the user-friendliness of bioinformatics tools by providing a web-based graphical user interface to commandline unix tools. Analysis pipelines often generate a large number of outputs, and with the explosion of data in next generation sequencing field, many of these resulting outputs contain millions of lines and can no longer easily be reviewed manually. Visualisation and summarisation of results has become an integral part of any analysis pipeline. The iReport tool provides generic reporting within the Galaxy framework, and Circos offers visualisation for high-dementional data.

\begin{comment}
Contents of this chapter:
\begin{enumerate}
\itemsep-0.5em
\item The Galaxy platform for accessible, reprodu- cible and collaborative biomedical analyses: 2018 update
\item iReport: A generalised Galaxy solution for integrated experimental reporting
\item Galactic Circos: Visualisation of high-dimensional data in Galaxy
\end{enumerate}
\end{comment}

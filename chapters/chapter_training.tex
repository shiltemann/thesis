\cleartorightpage
\begin{savequote}[75mm]
``We ignore public understanding of science at our peril''
\qauthor{Eugenie Clark}
\end{savequote}

\chapter{Training}\label{chapter:training}

\begin{figure}[t!]
\includegraphics[height=10em]{frontmatter/images/chapter-header-training.png}
\end{figure}
\setcounter{figure}{-1}
\setcounter{table}{-1}
\setcounter{section}{-1}

Training is a vital component of accessible research. Galaxy enables domain experts to perform complex analyses without programming or command line expertise. This user-friendliness also makes Galaxy an ideal platform for training, allowing learners to focus solely on the analysis and scientific concepts, rather than the minutiae of tool installation and maintenance or the command line interface details. To this end, we developed a central, community-driven infrastructure for Galaxy training materials, designed for ease-of-use for both trainees and instructors. Furthermore, by centralising training materials, maintenance and expansion becomes a collaborative effort supported by the global Galaxy trainer community.

This chapter contains the following sub-chapters:

\begin{enumerate}[label=\ref{chapter:training}.\arabic*]
\itemsep-0.5em
\setcounter{enumi}{-1}
\item \textbf{Community-Driven Data Analysis Training for Biology.} Together with Bérénice Batut (and later also joined by Helena Rasche), I founded the Galaxy Training Materials (\url{https://training.galaxyproject.org}) project in 2016. This involved creating the entire technical framework from the ground up. The web framework utilises Jekyll templating to automatically generate a website from Markdown documents. This allows for the creation of modern webpages for the tutorials, while allowing tutorial authors to create content in simple Markdown documents. Furthermore, we created the infrastructure needed for the tutorials to be self-contained (e.g. everything needed to follow them is freely and openly available online) and FAIR (e.g. datasets available in Zenodo, automatic BioSchemas annotation for findability in e.g. the TeSS training portal). Through years of active community building efforts, we have grown this into a mature FAIR training platform, currently featuring nearly 200 tutorials across 21 topics, authored by over 170 contributors (\url{https://training.galaxyproject.org/stats}). The tutorials typically recreate a published analysis from a scientific journal, further increasing accessibility of the bioinformatics pipelines. These tutorials are also being widely used by teachers around the world, both for bioinformatics workshops for scientists, as well as in higher education curricula. Notably this has also proven to be an invaluable training resource during the current COVID-19 pandemic; due to our focus on FAIR-ness of the materials, our tutorials could be easily used for virtual training events as well as in-person training, exemplified by the recent \emph{GTN Smörgäsbord} training event I organized with Helena Rasche for almost 1200 participants from 78 countries (\url{https://gallantries.github.io/posts/2021/03/01/sm\%C3\%B6rg\%C3\%A5sbord/}). This chapter is without a doubt the work I am most proud of in this thesis, and continues to be so as we are constantly working to enhance the GTN training platform and grow the community around it.
\end{enumerate}

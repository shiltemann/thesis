\chapter{Discussion}
\label{discussion}
\setcounter{figure}{-1}
\setcounter{table}{-1}
\setcounter{section}{-1}
\setcounter{NAT@ctr}{-1}

Write awesome discussion here!

Discuss current limitations and possible future remedies/directions

\section{The Bio, Futuromics}

\subsection{Single-cell Sequencing}

\subsection{Ultra-long high-quality reads}
making de-novo assembly possible

\subsection{Environment}
do not consider cancer cells in isolation but combine with surrounding tissue/immune system information and consider as a whole


\section{The Informatics}

<bigger better more RAM/CPU to keep up with the ever increasing data sizes>

<make it usable - the biologists have the knowledge for interpretation, software should be usable by them, not just bioinformaticians, Galaxy/docker/conda etc helps with this>

<make it maintainable as developers by adhering to coding best-practices and through the use of continuous integration strategies>

<training is vital, the more complex these methods get, the harder it will be to draw accurate conclusions>

<quantum computing?>


\section{The state of Academia}

The \textit{publish or perish} attitude in academia is bad; too many tools not maintained after first publication because a bioinformaticians worth seems to be determined solely by the number of first- or last-authorships of publications and the amount of funds obtained through grants. This leads a to huge replication of efforts, with many groups developing software to solve the exact same problem just because either the existing tools were inaccessible, or analysis not reproducible and not maintained, in large part because most bioinformaticians working in academia do not get time allocated for such tasks.

paywalled journals need to die

peer review process should also be open (anonymous upon request, but open)

do we even need journals?


just testing a citation: iReport \cite{hiltemann2014ireport}.

\bibliographystyle{ieeetr}
\bibliography{references}

\begin{savequote}[75mm]
``Humans are allergic to change. They love to say, `We've always done it this way.'' I try to fight that. That's why I have a clock on my wall that runs counter-clockwise.''
\qauthor{Grace Hopper}
\end{savequote}

\chapter{Discussion}\label{discussion}
\setcounter{figure}{-1}
\setcounter{table}{-1}
\setcounter{section}{-1}
\setcounter{NAT@ctr}{-1}

Since the completion of the human reference genome in 2003, the field of molecular biology has been transformed almost beyond recognition, and along with it, the field of bioinformatics has evolved from a niche discipline to an integral part of every biomolecular research question. Where just a couple decades ago most genomic data analyses could largely be performed by hand, nowadays they often require a supercomputer. Programming knowledge is required to run such analyses, but biologists are not typically trained in these skills, and have thus become reliant on bioinformaticians to carry out this work for them. Similarly, bioinformaticians often lack the increasingly complex biological knowledge required to fully interpret the results. Biologist and bioinformatician must therefore work together closely, and each must be trained in the other discipline sufficiently enough to be aware of any factors that might influence analysis or interpretation. Furthermore, data is being generated at an exponential rate (and bioinformaticians are not), and one way to address this gap is to empower researchers to run their own day-to-day analyses after analysis pipelines are fully developed and validated.


\section{Galaxy for open and accessible research}
The Galaxy project~\cite{TODO} is a framework that enables such empowerment of researchers to run complex data analysis without programming expertise. The Galaxy project is free and open-source and community-driven, solliciting feedback and code contributions from its user community to improve and evolve along with the ever-changing landscape that is bioinformatics. \hyperref[chapter:galaxy]{\textbf{Chapter~\ref{chapter:general}}} outlines the ongoing development in the Galaxy framework. Not only are new tools continually integrated by the community, new components are incorporated into the Galaxy code base and existing features improved to evolve and scale with user requirements. On the user-facing side, improvements were made to assist with the handling of large datasets (collections, rule-based uploader) to match the exponential rate of data generation. Accessibility was improved by the integration of a new help forum and the development of a central training materials repository (\textbf{Chapter~\ref{chapter:training}}).


\section{Training}
Training is an essential component in the dissemination of accessible bioinformatics tools and workflows. Training is needed both for researchers looking to analyze their data, as well as for the bioinformaticians developing the tools and workflows. Galaxy is especially well-suited for bioinformatics training because it provides a layer of abstraction, allowing trainees to focus on the bioinformatics \emph{concepts} rather than the nitty-gritty details about the UNIX commandline interface. Take for example a workshop on sequence mapping; when using Galaxy for teaching, the focus can lie purely on the different parameters and their effects on downstream analysis, whereas when teaching the same steps on the commandline, participants would have to learn new details about the UNIX commandline in addition to read mapping. Such an approad increases the cognitive load of students, which may hamper learning~\cite{paas2003cognitive}.

The Galaxy Training Network (GTN) is a loosely define group of instructors around the world who use Galaxy for training purposes, but there was little coordination between the different instructors in terms of materials used, and thus a lot of duplication of effort. Therefore, we set out to create a central repository of Galaxy training materials and a set of best-practice guidelines which could be used by anybody in the community. Any central solution would have to be a collaborative effort if it was to be sustainable and maintainable. In Chapter~\ref{chapter:training} we describe the collaborative web framework for delivery of bioinformatics training using the Galaxy platform that we created in response to this need.

We focused on creating a fully open and transparent framework that is accessible and easy to use for both learners and trainers. All development happens on GitHub, where anybody may suggest additions and changes, and any such proposed changes are thoroughly tested using Travis continuous integration system~\cite{travis-ci} to ensure functionality and adherence to guidelines. The proposed changes or additions to tutorials or the framework are then reviewed by one or more of the volunteers from the community, and changes may be requested. Once approved, the code is merged into the main code base, and the new website is automatically built and deployed by Travis and GitHub. The materials themselves are based around \emph{research stories}; usually the recreation of results described in a published paper. This gives users the confidence that the tools and pipelines they are learning are practically useful and publication-level quality, as well providing them with the opportunity to explore the science and informatics in the paper in full detail. Furthermore, since publication of our work, a number of publications of scientific papers have included Galaxy training materials for readers and as a form of documentation~\cite{TODO}.

One of the main challenges in designing this framework, was to allow easy contributions from instructors, without the need for any web development knowledge. To this end, we used Jekyll templating~\cite{jekyll}, which allows tutorials to be written in the simple and accessible markup language Markdown~\cite{markdown}, which is then automatically converted to a web page. Analogous to how Galaxy allows scientists to run analyses while being abstracted away from the implementation layer of the tools, so does jekyll allow instructors to create web pages for their tutorials without being concerned with the web application layer.

A further challenge was to enable the materials to be usable both by instructors during workshops, as by individuals learning on their own. This is accomplished by including all materials instructors might provide during a workshop in the GTN training materials framework. This includes slides as well as hand-on materials, further reading suggestions, and an automatically created and updated list of Galaxy servers which meet the requirements to run a given tutorial.

\begin{comment}
<closer integration with galaxy servers>

<feedback, planemo, cofests, dashboard, instructor topic, levels, translations, tess search, curricula, toc>

2019: 4 new topics (metabolomics, computational chemistry, data manipulation, ui and features), 66 new tutorials
\end{comment}

Since the initial publication of Chapter~\ref{chapter:training}, we have continued active development of the GTN framework and training materials. As Galaxy evolves, so must the associated tutorials; where Galaxy is expanding beyond bioinformatics and is now also being used in fields such as natural language processing and computational chemistry, so have we noticed a steady expansion of topics and tutorials contributed by the community. In the year following publication, we saw 6 new topics added, 66 new tutorials, and the number of contributors grew from 64 to 137. Where the focus of development initially lay with improving the experience for end-users of the tutorials, in the year since publication, our focus has shifted to support for tutorial contributors and instructors intending to use our materials.

In order to support this growing and diverse community of learners and trainers, constant maintenance and development is required. The main challenge in the coming years will be the community management; creating and sustaining a close-knit community of Galaxy users and instructors so that the project can survive even when its original developers have moved on.



\section{Visualisation and Reporting}

Galaxy is highly flexible, but this comes hand-in-hand with complexity. This trade-off is acceptable for researchers still developing their pipelines, but for clinicians who have fixed and validated workflows, the flexibility is not longer required, and in many cases even undesirable. At this point the relative complexity of the Galaxy interface to novice users may become a hurdle.

One of the main deficits of Galaxy in its current form, is the lack of an appealing system of results reporting. Datasets produced from workflows must be individually viewed by the user. While Galaxy sports a plugin system for visualisation of individual datasets, a generic reporting tool for displaying a set of output datasets together does not exist. To this end, we developed iReport (Chapter~\ref{chapter:general}), a fully customizable Galaxy tool that generates a webpage that can display any number of workflow outputs. In this manner, end-users need only to open a single workflow output to get a full overview of results. The workflow developer can create an iReport fully tailored to the needs of their users, including the ability to add links to datasets and external resources, create searchable and sortable tables, add images and custom text, and to divide content into different pages. In this manner, clinicians and other end users are able to run workflows and view results with minimal instruction and knowledge of the pipelines and Galaxy interface.

Going even further, by using the Galaxy API, simplified web frontends for Galaxy may be created, such that only features needed by the users are exposed. In the case of clinical applications, this provides the additional benefit of protecting user from inadvertently altering any workflow parameters. An example of such a simplified front-end is Galaksio~\cite{galaksio} or IRIDA~\cite{}. Users log in to a custom webpage, are given a list of workflow to choose from, and are asked to provide input datasets. While Galaxy is used to run the analysis, the users never leave their web environment, and everything is handled by the front-end. In order to fully


\section{Use Cases}

The concepts described in the previous sections were applied to two separate use cases. Chapters~\ref{chapter:fusiongenes} and~\ref{chapter:variants} describes the creation of analysis tools and pipelines for (prostate) cancer analysis, and in Chapter~\ref{chapter:microbiota}, Galaxy-based analysis pipelines were built and tested for microbiota profiling. While from a biological perspective these application fields are quite distinct, from a bioinformatics perspective they share many similarities. Both involve genome sequencing data, which have to be quality checked and mapped to a reference sequence or database. After these common steps the pipelines start to diverge more significantly; where the cancer pipelines focus on variant detection, microbiota profiling is concerned with the identification of the microbial entities present. Both have the ultimate end-goal of improving understanding of the biological mechanisms involved, and facilitating the diagnosis and treatment of patients. To that end, the incorporation of steps to visualize the large result files and reporting the various pipeline outcomes in a unified report are again shared steps between both use cases. Where the prostate cancer pipelines were developed mainly for research purposes, the microbiota pipeline was developed for direct clinical use, posing additional challenges in terms of accuracy and robustness.


\subsection{Cancer Analysis}
ifuse and vcap

GmT and virtual normal


\subsection{Microbiota Analysis}
The MYcrobiota project described in Chapter \ref{chapter:microbiota} was aimed at developing a 16S microbiota profiling pipeline suitable for use in a clinical setting, in order to augment or eventually potentially replace the culture-based methods currently employed in microbial diagnostic laboratories. While 16S sequencing is a relatively well-established technique, there are several obstacles yet to overcome to enable its use in routine diagnostics. These obstacles include 1) the high prevalence of chimera formation during PCR amplification, 2) the inability to standardize the relative abundance results obtained from 16S profiling across different studies, and 3) the lack of a user-friendly bioinformatics pipeline that can be operated by clinicians without extensive bioinformatics knowledge.

The MYcrobiota platform is the result of a close collaboration with Streeklab Haarlem \cite{streeklab url}, a microbial diagnostics lab servicing a large number of GPs and hospitals in the region. In order to facilitate the use of 16S rRNA sequencing in a diagnostic setting, several enhancements to standard procedure were required, both in the wetlab and the bioinformatics pipelines to overcome the aforementioned obstacles. The MYcrobiota platform utilizes the micelle PCR (micPCR) method \cite{boers2015micelle,boers2017novel}. In this approach, PCR amplifications of each template 16S template sequence occurs in a physically distinct reaction environment, thus greatly reducing the generation of hybrid sequences known as chimears. This approach has the additional benefit of

Several obstacles to its use in clinic, chimera formation,  absolute vs relative abundance (solved by micelle) and a simplified bioinformatics pipelline that can b operated by non-bioinformaticians (galaxy)

several strategies (optimized PCR protocols to prevent detection and complex algorithms to remove during analysis) to reduce chimera formation, but none capable of preventing them altogether~\cite{huttenhower2012structure}. micPCR to prevent chimera formation, and reduce PCR competition induced bias

proportional abundances are the standard, but this complicates cross-study comparisons, micPCR approach allows for absolute quantification when used in conjunction is an internal calibrator "This IC consists of quantified genomic DNA from a bacterium that is selected for its absence in the natural microbial flora of the investigated samples and is added to each DNA extract prior to micPCR amplification" (quoted from stephans thesis) - used to convert  the obtained 16S reads per OTU to 16S gene copies per OTU

samples sequenced in triplicate to "average out any possible quantification bias generated due to differences in the distribution of micelle sizes between independent micPCR experiments"

remove contamination by excluding sequences not reproducably measured in triplicate, and those measured in triplicate in a negative extraction control sample

clinical validation: (from stephan) "the application of MYcrobiate for routine clinical microbiological diagnostics use was evaluaed by investigating a total of 63 clinical samples, including 40 (polymicrobial) clinical samples obtained from patients presenting a variety of damaged skin conditions (ref our science paper) and 23 low biomass clinical samples obtained from patients who were suspected to have bacterial septic arthritis.The results were compared to the results opbatined with culture-based methods, which are the current gold standard methods for pathogen detection in the routine clinical microbiological diagnostic laboratory.
As shown in <our paper>, 36 of 38 aerobic bacteria identified within the damaged skin samples using routine culturing methods were alsi identified using mycrobiota, altho the majority of the 447 bactiral taxa identified using myrobiota were presumed to belong to the commensal flora or were not cultured at all. These results indicate dthat the resolution power of mycrobiota was superior compared th culture based methods"..detects anaerobic microbes not culturable "importantly, anaerobic bacteria are a common cause of endogenous bacterial infections and their cultur-free detection by mycrobiota would provide clinitions with very suseful information about the aetiologies of such infections that cannot be (easily) provided using routine culturing methods"

"it should be noted however that the partial 16S rRNA genes which are currently targeted by MYCrobiota lack the discriminative power to differentiate prokaryotes to the species taxonomic level \cite{konstantinidis2007prokaryotic} This species-level  determination is often seen as essiential for clinical diagnostics, as only specific species within a genus may be pathogenie. Importantly however, the identification  of bacterial species using species-secific qPCR is one wat to circumvent this limitation while other strategies would require relatively simple adjustments to be made to the mucrobiota platform that enables the micelle-based amplifications, sequencein and analysis of multiple hypervariable 16S rRNA gene regions \cite{jumpstart2012evaluation} or other genetic markers such as rpoB~\cite{adekambi2009rpob}, gyrB~\cite{yanamoto1995pcr}, the ITS region~\cite{schoch20012nuclear}, and many other candidates that enable taxonomic differentiation at the species level \cite{lab2016marker,sabat2017targeted}"

i
"The ability of MYCrobiota to remove contaminating DNA allows the accurate detection of potentially pathogenic microorganisms at very low abundances, or alternatively, the confirmation of culture-negative results (confirmed in study from chap7 from stephan, arthritis)"


"these findings indicate that mycrobiota is a very useful platform that enables the culture-free detection of anaerobic, fastidious and (unexpected) culturable microorganisms, which will greately improve the identification of the bacteriological aetiology of infections such as bacterial septic arhtritis. However, extensive clinical validation studies will be needed in order to validate the routine introduction of MYcrobiota into clinical diagnostic laboratories"

universality of pipeline tested, pipe;ine used in environmental applications in drinking water distribution systems, and detected spatial and temporal microbial variations not detected using routine culturing methods "Monitoring of microbial dynamics in a drinking water distribution system using the culture-free, user-friendly, MYcrobiota platform" Sci Rep 2018 (stephans paper) and "detection of bacterial DNA in septic arthritis samples using the MYcrobiota platform" - J clin Rheumatol 2018 (stephans other paper)


\section{Future Perspectives}

Galaxy: reporting, bigger data, other application, IDC, shared reference data,

\subsection{The Bio: Futuromics}
single cells, real-time, bigger data

microbiota: long reads to improve shotgun analysis, or 16S higher accuracy

\subsection{The Informatics}

Galaxy in the clinic: reporting, frontends

Open Science! Open Science! Open Science!

\bibliographystyle{ieeetr}
\bibliography{references}

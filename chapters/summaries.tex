\begin{savequote}[75mm]
``Werken en feesten vormt schoone geesten.''
\qauthor{Johanna Westerdijk}
\end{savequote}

\chapter{Summary}\label{chapter:summary}
\setcounter{figure}{-1}
\setcounter{table}{-1}
\setcounter{section}{-1}
\setcounter{NAT@ctr}{-1}

\begin{figure}[t!]
\includegraphics[height=10em]{frontmatter/images/samenvatting.png}
\end{figure}

\begin{enumerate}[label=\ref{chapter:summary}.\arabic*]
\itemsep-0.5em
\item Summary (English)
\item Summary (Dutch)
\end{enumerate}


\chapter*{Summary}
\phantomsection\addcontentsline{toc}{section}{Summary}

Recent technological advances in the area of genomic sequencing have resulted in an explosion of the amount of data being generated. As a result, the life sciences have become increasingly computational in nature, and bioinformatics has taken on a central role in biomedical research. However, the analysis of these datasets typically requires specialist bioinformatics skills to run. This can lead to bioinformatics data analysis appearing like a mystical \textit{black box} to the researchers and clinicians tasked with interpreting the results. However, a basic understanding of the tools and processes that make up the analysis pipelines is crucial for accurate understanding of results. In this thesis, we aim to shine a light into this black box; to de-mystify bioinformatics pipelines, and empower researchers and clinician to run their own data analyses.

Chapter \ref{chapter:introduction} provides the relevant background about genomics, sequencing, and bioinformatics best-practices.

In order to make bioinformatics analyses more accessible to the domain experts, several things are required.


## TECHNICAL (GAlaxy, Circos, iReport)
\textbf{Chapter} \ref{chapter:introduction} provides a general introduction to genomics, bioinformatics,

Chapter \ref{chapter:technical} covers some of the technical groundwork required for making bioinformatics accessible to researchers and Clinicians.

- Galaxy
- Circos
- iReport

## v
Analysis tools and workflows are Chapter \ref{chapter:training}

## Fusion Genes

## Somatic Variant Detection

## Microbiota profiling


## Open Science, conclusion


\chapter*{Samenvatting}
\phantomsection\addcontentsline{toc}{section}{Samenvatting}

samenvatting..

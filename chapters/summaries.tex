\begin{savequote}[75mm]
``Werken en feesten vormt schoone geesten.''
\qauthor{Johanna Westerdijk}
\end{savequote}

\chapter{Summary}\label{chapter:summary}
\setcounter{figure}{-1}
\setcounter{table}{-1}
\setcounter{section}{-1}
\setcounter{NAT@ctr}{-1}

\begin{figure}[t!]
\includegraphics[height=10em]{frontmatter/images/samenvatting.png}
\end{figure}

\begin{enumerate}[label=\ref{chapter:summary}.\arabic*]
\itemsep-0.5em
\setcounter{enumi}{-1}
\item Summary (English)
\item Summary (Dutch)
\end{enumerate}


\section{Summary}
\phantomsection\addcontentsline{toc}{section}{Summary}

DNA is often referred to as the \emph{source code of life}; it encodes the proteins that control the functioning of our cells and plays a huge role in our health. The publication of the human reference genome in 2003, combined with sustained technological advances in genome sequencing ever since, have transformed the field biomedical research, and have led to an explosion of the amount of data being generated.
However, scientists typically aren't trained in the skills required to manage and analyse these large datasets. Furthermore, bioinformatics tools and workflows tend to be very complex, and often require programming skills to run. As a result, researchers often rely on bioinformaticians to perform the data analyses for them.
This skills gap can lead to the bioinformatics data analysis feeling like a mystical \textit{black box} to the researchers and clinicians tasked with interpreting the data analysis results.
However, a basic understanding of the tools and processes that make up the analysis pipelines is often crucial for an accurate understanding of the analysis results. In this thesis, we aim to shine a light into this black box; to de-mystify bioinformatics, and increase the accessibility of tools and workflows in order to empower researchers and clinician to run their own data analyses.
To this end, we first developed the required technical framework, both for the data analysis itself, and for the training of researchers and clinicians in the use of bioinformatics tools and workflows. We then proceeded to applied this framework and the Open Science methodology to a set of use cases, in prostate cancer analysis and microbiota profiling.

\textbf{Chapter \ref{introduction}} gives a short introduction to genomics, including a brief history of genome sequencing. It also describes the bioinformatics challenges faced in the analysis of these often large and complex datasets, and provides a set of best-practice guidelines to facilitate accessibility, reprodicubility and interoperability of bioinformatics processes. Finally, it provides a short background for each of the use cases, including fusion genes in prostate cancer, and microbiota profiling.

In order to make bioinformatics analyses more accessible to the domain experts, some technical groundwork is required. This is provided in
\textbf{Chapter \ref{chapter:general}}. Here we introduce the Galaxy platform, a web-based bioinformatics workflow engine that enables scientist to analyse large datasets with nothing more than a web browser.
Once data has been analysed, it must be presented in a comprehensible way to the researchers and clinicians capable of interpreting the results. Efficient visualisation of data is crucial here.
Therefore, as part of this chapter, we have integrated the Circos tool into the Galaxy framework. This tool allows us to plot genome-wide data in a single circular plot. By using Galaxy, this tool becomes accessible to non-bioinformaticians as well.
Finally, we developed iReport, a tool within the Galaxy framework that allows the creation of custom web-reports for results reporting. This tool can combine results from any number of tools, and may be tailored to fit the user's need. Together, these components form the basis for brining bioinformatics analysis to biomedical researchers and clinicians.

With the technical framework in place, the next important step is the training of the researchers and clinicians in the use of this platform, as well as in the relevant computational an informatics concepts that may impact interpretation of analysis results.
To this end, in \textbf{chapter \ref{chapter:training}}, we developed the Galaxy training repository, in close collaboration with the University of Freiburg. This project created a central repository for Galaxy-based training materials. It is a inherently community-driven project, and we put great effort into facilitating contributions from researchers and teachers around the globe.

Next, we applied this process for accessible bioinformatics to a number of use cases. In \textbf{chapter \ref{chapter:fusiongenes}} we examined fusion genes in a prostate cancer cell-line. First, we developed iFUSE, a web-pased application for the visualisation and exploration of potential fusion genes. This application was used to identify a large number of fusion genes in the VCaP cell-line, and prioritize them on basis of confidence and impact for further confirmation. Futhermore, we determined that one of the arms of chromosome 5 had undergone chromothripsis. This is a shattering of the genome in a single catastrophic event, and the subsequent imprecise stitching back together of the chromosomes by the cell's repair mechanisms. Using the Circos tool we were able to visualize this very effectively.
A second use case within the prostate cancer domain is described in \textbf{chapter \ref{chapter:virtualnormal}}. When sequencing tumour samples, typically a sample from healthy tissue of the same patient is also sequenced, in order to distinguish the cancer-specific (somatic) variants from the germline variants. However, such an associated normal sample is not always available. For these cases, we examined the feasibility of using a \emph{virtual normal} instead. This is a set of healthy genomes from healthy, unrelated, genetically diverse individuals. To this end, we first integrated a suite tools for variant analysis and visualisation into Galaxy, and combined them into a workflow.

In additionion to these two research-oriented use cases, \textbf{chapter \ref{chapter:microbiota}} describes a clinical use case. Here we developed the MYcrobiota platform for diagnostic microbiota profiling using 16S sequencing. This pipeline was developed for use by Streeklab Haarlem as an augmentation to their clinical diagnostic pracitices when traditional methods do not provide a clear answer.
First, we integrated the full mothur suite of 125+ tools into Galaxy, and then combined them into a workflow. In order to accomodate the specific experimental setup employed by the Streeklab, we additionally developed a set of auxiliary Galaxy tools to supplement the standard analysis pipeline.
These tools, as well as several preexisting Galaxy tools were integrated into an end-to-end workflow, with an iReport at the end for results reporting. This workflow was carefully tested and validated in collaboration with the Streeklab. We also developed Docker images with the full Galaxy setup so that it can be run in-house by the Streeklab.

For each of these use cases, we adhered to the bioinformatics best-practices and Open Science principles outlined in the introduction, and all the code is freely available in GitHub. Training materials have also been developed in order to aid not only our own researchers and clinicians in the use of the tools and workflows we developed, but also others around the world. All these training materials have deposited in the Galaxy training repository described in chapter \ref{chapter:training}.


\section{Samenvatting}
\phantomsection\addcontentsline{toc}{section}{Samenvatting}

[Will translate once the english version has been approved and finallized]

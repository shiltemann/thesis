\chapter{Glossary of terms}
\label{AppendixA}

\textbf{\$1,000 genome} In October 2006, the X Prize Foundation, working in collaboration with the J. Craig Venter Science Foundation, established the Archon X Prize for Genomics, intending to award \$10 million to ``the first team that can build a device and use it to sequence 100 human genomes within 10 days or less, with an accuracy of no more than one error in every 1,000,000 bases sequenced, with sequences accurately covering at least 98\% of the genome, and at a recurring cost of no more than \$1,000 per genome''.

\textbf{Aneuploidy} is the presence of an abnormal number of chromosomes in a cell, for example a human cell having 45 or 47 chromosomes instead of the usual 46. It does not include a difference of one or more complete sets of chromosomes, which is called euploidy.

\textbf{Angiogenesis} is the physiological process through which new blood vessels form from pre-existing vessels.

\textbf{BCR-ABL} oncogene found on the Philadelphia Chromosome, a piece of genetic material seen in Chronic Myelogenous Leukemia caused by the translocation of pieces from chromosomes 9 and 22. Bcr-Abl codes for a tyrosine kinase, which is constitutively active, leading to uncontrolled cell proliferation.

\textbf{Carcinogenesis} is the accumulation of multiple genetic alterations that drive a normal cell to malignancy

\textbf{End Sequence Profiling (ESP)} see paired-end mapping.

\textbf{Extracellular Matrix} (ECM) is a collection of extracellular molecules secreted by cells that provides structural and biochemical support to the surrounding cells

\textbf{GWAS} or genome-wide association study, also known as whole genome association study (WGAS), is an examination of a genome-wide set of genetic variants in different individuals to see if any variant is associated with a trait

\textbf{Integrins} are transmembrane receptors that facilitate cell-extracellular matrix (ECM) adhesion. Upon ligand binding, integrins activate signal transduction pathways that mediate cellular signals

\textbf{LOH} stands for loss of heterozygosity, and indicates an event that results in loss of the entire gene and the surrounding chromosomal region

\textbf{MYC} the MYC gene is implicated in Burkitt's Lymphoma, which starts when a chromosomal translocation moves an enhancer sequence within the vicinity of the MYC gene. The MYC gene codes for widely used transcription factors. When the enhancer sequence is wrongly placed, these transcription factors are produced at much higher rates.

\textbf{Oncogene} are mutated forms of normal cellular genes generally involved in promoting cell proliferation.  These mutations result in dominant gain of function.

\textbf{Paired-end sequencing (PEM)}

\textbf{Proto-oncogene} is a normal gene that could become an oncogene due to mutations or increased expression. Examples of proto-oncogenes include RAS, WNT, MYC, ERK, and TRK

\textbf{RAS} is a family of related proteins which is expressed in all animal cell lineages and organs, When Ras is activated by incoming signals, it subsequently switches on other proteins, which ultimately turn on genes involved in cell growth, differentiation and survival. Mutations in RAS genes can lead to the production of permanently activated Ras proteins. As a result, this can cause unintended and overactive signaling inside the cell, even in the absence of incoming signals.

\textbf{SNV} Single Nucleotide Variant, a variant consisting of a mutation of a single nucleotide relative to the reference genome.

\textbf{Stromal cells} are connective tissue cells of any organ, for example in the uterine mucosa (endometrium), prostate, bone marrow, lymph node and the ovary. They are cells that support the function of the parenchymal cells of that organ. *Fibroblasts* and *pericytes* are among the most common types of stromal cells.

\textbf{Transcriptome} TODO: this definition was copied from paper, paraphrase -- the complete set of transcripts in a cell, and their quantity, for a specific developmental stage or physiological condition. Understanding the transcriptome is essential for interpreting the functional elements of the genome and revealing the molecular constituents of cells and tissues, and also for understanding development and disease. The key aims of transcriptomics are: to catalogue all species of transcript, including mRNAs, non-coding RNAs and small RNAs; to determine the transcriptional structure of genes, in terms of their start sites, 5′ and 3′ ends, splicing patterns and other post-transcriptional modifications; and to quantify the changing expression levels of each transcript during development and under different conditions.

\textbf{Tumour Suppressor genes} are genes whose normal function in regulating proliferation is to stop it. Mutation results in recessive loss of function.

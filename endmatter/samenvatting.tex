\chapter{Samenvatting}

DNA word vaak beschouwd als de \emph{broncode van het leven}; het encodeert
de eiwitten die onze celprocessen beinvloeden, en speelt een belangrijke rol in onze gezondheid.
De publicatie van het humane referentie genoom in 2003, in combinatie met continue technologische vooruitgangen sindsdien, hebben het veld van biomedisch onderzoek volkomen getransformeerd, en hebben geresulteerd in een explosie van de hoeveelheid data die gegenereerd wordt in de biomedische wetenschappen.

Wetenschappers worden echter meestal niet opgeleid in de benodigde vaardigheden om deze efficient om te gaan met deze grote hoeveelheden complexe datasets en ze te analyseren.
Verder zijn bioinformatica tools en pijplijnen vaak erg complex, en zijn programmeervaardigheden meestal benodigd om ze te kunnen gebruiken. Hierdoor onstaat de situatie waarin onderzoekers vaak afhankelijk zijn van gespecialiseerde bioinformatici om hun analyses uit te voeren.
Door deze vaardigheidskloof kan bioinformatica vaak aanvoelen als een \emph{black box} voor de onderzoekers en clinici die de resultaten van deze analyse moeten interpreteren.
Desalniettemin is een basiskennis van de achterliggende computationele concepten vaak essentieel voor de accurate interpretatie van de resultaten.
In dit proefschrift trachtten wij de bioinformatische \emph{black box} te illumineren, en de tools en pijplijnen toegankelijk te maken voor onderzoekers en clinici, zodat ze hun eigen data analyses weer uit kunnen voeren, zonder afhankelijk te zijn van tussenkomst van een bioinformaticus.

Hiervoor hebben we eerst de technische grondslag ontwikkeld, zowel voor het uitvoeren van de data analyses, alsmede het opleiden van onderzoekers en clinici in het gebruik ervan. Vervolgens hebben we dit technische framework toegepast via een set van wetenschappelijke casussen, in het gebied van prostaatkanker en micribiota analyse.

\textbf{Hoofdstuk~\ref{introduction}} geeft een korte algemene introductie tot genomics, waaronder een korte geschiedenis van sequencing technologieen.
Hierbij worden ook de uitdagingen beschreven die we tegenkomen in de bioinformatica bij het analyseren van de resulterende grote en complexe datasets, en geven we een set richtlijnen voor het faciliteren van toegankelijke, reproduceerbare en interoperabele data analyse.
Tot slot wordt hier een korte achtergrond gegeven voor ieder van de casussen.

\textbf{Hoofdstuk~\ref{chapter:general}} beschrijft de technische grondslag die we ontwikkeld hebben om de bioinformatica meer toegankelijk te maken voor de wetenschappelijke experts die de data genereren en de resultaten zullen interpreteren.
Hier introduceren we allereerst Galaxy, een gebruiksvriendelijk data analyse platform waarmee wetenschappers makkelijk hun datasets kunnen analyseren met enkel een web browser. Na het analyseren van de data moeten de resultaten op een gestructureerde en overzichtelijke manier gepresenteerd worden aan de gebruiker. Visualizatie is hierbij cruciaal.
Daarom hebben we als tweede deel van dit hoofdstuk de Circos tool in Galaxy geintegreerd. Circos is een krachtige tool waarmee we genoom-wijde data efficient in een circulair plot kunnen weergeven.
Normaalgesproken vergt deze tool gespecialiseerde kennis van bioinformatica, maar door deze binnen Galaxy beschikbaar te maken kan de tool ook gebruikt worden zonder deze technische kennis.
Tot slot beschrijft dit hoofdstuk de iReport tool, waarmee gebruikers zelf web reports kunnen configureren binnen Galaxy om de resultaten van hun analyses weer te geven, wat precies afgestemd kan woorden op hun behoeften.
Samen zorgen deze 3 componenten ervoor dat NGS analyses toegankelijk worden voor biomedische onderzoekers en clinici.

Met de technische grondslag gelegd, is de volgende belangrijke stap het opleiden van onderzoekers
en clinici om hiermee om te gaan, zowel als de benodigde basiskennis op te doen over de achterliggende computationele concepten die een effect kunnen hebben op de interpretatie van de resultaten.
In \textbf{Hoofdstuk~\ref{chapter:training}} hebben we een de Galaxy Training Repository ontwikkeld, in nauwe samenwerking met de Universiteit van Freiburg.
In dit project hebben we een centrale repository ontwikkeld voor het verzamelen en onderhouden van wetenschappelijke training materialen die begruik maken van het Galaxy platform.
Door de grootschalige aard van dit project, is het zo opgezet dat gebruikersgemeenschap rond Galaxy het project samen draaiend kan houden, en gebruik gemaakt kan worden van de gecombineerde expertise van wetenschappers en opleiders over de hele wereld.

Vervolgens hebben we deze aanpak toegepast op een aantal casussen. In \textbf{Hoofdstuk~\ref{chapter:fusiongenes}} hebben we fusie genen in een prostaatkanker cellijn onderzocht. Hiervoor hebben we eerst een applicatie ontwikkeld (\emph{iFUSE}) voor het visualiseren van structurele varianten en het identificeren van fusiegen kandidaten.
Door gebruik van deze applicatie, in combinatie met de Circos tool, waren we in staat een groot aantal potentiele fusie genen te identificeren in de VCaP cellijn, en hebben we deze bevindingen in het lab kunnen bevestigen.
Verder hebben we zo ook kunnen vaststellen dat de q arm van chromosoom 5 \emph{chromothripsis} vertoonde, een phenomeen waarbij een deel van het genoom verbrijzeld wordt, waarna het genetische materiaal vervolgens op inaccurate wijze weer aan elkaar geplakt wordt door de reparatiemechanismen van de cel.

Een tweede casus binnen het prostaat kanker domein beschrijven we in \textbf{Hoofdstuk~\ref{chapter:virtualnormal}}. Wanneer we een tumor sequencen, wordt er normaal gesproken meestal ook gezond weefsel van dezelfde patient gesequenced.
Hierdoor kunnen we de tumorcellen vergelijken met gezonde cellen, om zodoende de kanker-specifieke varianten te onderscheiden van de germline cellen.
Echter, zo'n geassocieerd normaal sample is niet altijd beschikbaar.
In zulke gevallen wilden we achterhalen of deze aanpak gesimuleerd kon worden door het gebruiken van een grote set van monsters afkomstig van gezonde, ongerelateerde, ethnisch diverse individuelen.
Hiervoor hebben we eerst een set analyse tools geidentificeerd en ontwikkeld, en deze in Galaxy beschikbaar gemaakt, en ze gecombineerd tot een pijplijn.

Naast deze twee onderzoeks georienteerde toepassingen, beschrijven we in \textbf{Hoofdstuk~\ref{chapter:microbiota}} een derde, clinisch georienteerde casus. Hier hebben we het MYcrobiota platform ontwikkeld voor de profilering van microbiota via 16S rRNA sequencing voor gebruik binnen de diagnostiek.
Deze applicatie is ontwikkeld in samenwerking met het Streeklab Haarlem, als aanvulling op hun traditionele diagnostiek, wanneer de resultaten daarvan geen uitsluitsel geven.

Hiervoor hebben we eerst de complete mothur toolkit van 125+ tools in Galaxy geintegreerd, en een workflow gemaakt die precies afgestemd is op de exacte experimentele setup van het Streeklab.
Hiervoor moesten ook een aantal nieuwe tools gescherven worden als aanvulling op de standaard analyse procedure.
Deze tools, in combinatie met reeds bestaande Galaxy tools, hebben we samengevoegd tot een pijplijn, met een iReport aan het eind voor het presenteren van de resultaten aan de clinici.
Deze pijplijn is grondig gtest en gevalideerd in samenwerking met de experts op het Streeklab, alvorens in gebruik genomen te worden.
Om het mogelijk te maken om MYcrobiota binnenshuis te draaien op het Streeklab, hebben we Docker images ontwikkeld met alle benodigdheden om de applicatie gemakkelijk overal te kunnen draaien.

Voor al deze toepassing hebben we ons gehouden aan de richtlijnen die we hebben beschreven in de introducite, en alle code is compleet open en voor iedereen vrij te gebruiken. Ook hebben we voor iedere casus training materialen ontwikkeld, en ondergebracht in de Galaxy training repository beschreven in \textbf{Hoofdstuk~\ref{chapter:training}}, waardoor ze niet alleen beschikbaar zijn voor onze eigen onderzoekers en clinici, maar voor de wereldwijde gebruikersgemeenschap.



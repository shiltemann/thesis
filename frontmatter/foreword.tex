% the abstract
\chapter*{Foreword}
\setlength\parindent{0pt}
\vspace{-1cm}
I like puzzles. Any type of puzzle. I always have. If I see a puzzle or a problem I have
to solve it. I think that is what makes cancer such a fascinating topic for me.

Imagine you are given a jigsaw puzzle. Now instead of a few hundred pieces, there are
several billion pieces. The picture on the box is not a picture of the puzzle inside the box,
it is just a somewhat similar image. Oh, and did I mention there are a whole bunch
of pieces missing? and that many pieces are duplicated? Some pieces don't even belong in our box,
but come from a completely different puzzle. On top of that your little sister has spilled paint
over some of the pieces so those can't be trusted to contribute to the image. And instead of one
single puzzle, the box contains several, they are all variations of the image on the box, but you have
no idea how many different puzzles the box contains.

Sound challenging? This is the problem we are solving whenever we sequence a cancer genome.

\vspace*{0.5cm}
\textbf{[Metaphor key]} \\
\textit{puzzle pieces} = sequence reads \\
\textit{picture on box} = reference genome \\
\textit{missing pieces} = hard-to-sequence areas \\
\textit{other puzzles} = contamination \\
\textit{painted pieces} = sequencing errors \\
\textit{multiple puzzles in box} = clonality

\verb+<thought: have the metaphor low-key run throughout thesis? >+

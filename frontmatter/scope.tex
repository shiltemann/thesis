\chapter*{Structure of Thesis}
\vspace{-2cm}
(TODO: remove, this is just a note for myself/reviewers)


\textbf{Title: } Bioinformatics for Everybody \\
\textit{Reasoning: }\vspace{-1.8em}
\begin{itemize}
\itemsep-0.5em
  \item \textbf{Galaxy:} empowers researchers to run their own analyses
  \item \textbf{Training materials infrastructure:} educates researchers to understand their own analyses
  \item \textbf{Visualisation and reporting:} for easy interpretation of results (e.g. for clinicians)
\end{itemize}
\vspace{-1.8em}
Focus on accessible, reproducible, user-friendly bioinformatics and open science principles.


\textbf{Publications:} \\

Chapter 1-3: The bioinformatics foundations \\
Chapters 4-6: Use cases; split into "the bio" and "the informatics"; Technical paper \& the biological insight it led to/facilitated \\

\begin{itemize}
\itemsep-0.5em
\item \textbf{Chapter 1:} Galaxy platform \\
\item \textbf{Chapter 2:} Training Materials infrastructure paper \\
\item \textbf{Chapter 3:} Visualization \& results reporting \\
  - iReport paper \\
  - Circos paper (hopefully)
\item \textbf{Chapter 4:} Use case: Fusion Gene Detection \\
  - the informatics: iFUSE paper \\
  - the bio: VCaP chromothripsis paper
\item \textbf{Chapter 5:} Use case: Somatic Variant Detection \\
  - the informatics: CGtag paper (Galaxy tool suite) \\
  - the bio: Virtual normal paper
\item \textbf{Chapter 6:} Use case: Micriobiota Profiling \\
  - the informatics: GmT paper (Galaxy tool suite \& training manual) \\
  - the bio: MYcrobiota platform
\end{itemize}

\textbf{Structure of Introduction}

\begin{itemize}
\item \textbf{The Source code of life} - General introduction to DNA and backgrund on genome sequencing
\item \textbf{The Bioinformatics Challenge} - Some challenges faced in bioinformatics analyses
\item \textbf{Bioinformatics Best Practices} - Some guidelines for high-quality bioinformatics tools and practices to help deal with challenges mentioned in previous paragraph
\item \textbf{Bioinformatics for Everybody} - Galaxy and Training materials as key component for delivering accessible, user-friendly bioinformatics analyses for non-bioinformaticians
\item \textbf{Use Case 1: Prostate Cancer} - Cancers complexities (Fusion genes, chromothripsis), somatic variant detection
\item \textbf{Use Case 2: Micriobiota profiling} - Intro to the microbiome, 16S rRNA sequencing vs whole-genome shotgun
\end{itemize}

\textbf{Structure of Discussion}

\begin{itemize}
\item \textbf{TODO}
\end{itemize}

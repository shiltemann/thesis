\pagenumbering{gobble}
\begin{center}
Propositions accompanying the thesis

\textbf{Jigsaw Genomics \ \\Assembling the pieces toward open and accessible bioinformatics for everyone}

by Saskia Hiltemann.
\end{center}



\begin{enumerate}
\item The Galaxy platform increases accessibility of bioinformatics analyses and enhances reproducibility of scientific research (this thesis).
\item High-quality bioinformatics training is an integral part of empowering domain experts to analyze their own data (this thesis).
\item iFUSE and Circos visualizations were instrumental in identifying fusion genes and chromothripsis in the VCaP cell line (this thesis).
\item A large, representative virtual normal set is equally valuable as an associated normal sample for germline correction of tumour samples (this thesis).
\item The MYcrobiota platform improves standardization and reproducibility across experiments, studies, and laboratories for microbiota profiling in diagnostics (this thesis).
\item The COVID-19 pandemic has illustrated the power of, and need for Open Science (10.1101/2020.08.13.249847).
\item Efforts to attract more women to STEM (Science, Technology, Engineering and Mathematics) should target parents, teachers and the general public, not only young women (10.1007/s11199-011-9996-2).
\item More effective communication of scientific results to the general public, and in particular the robustness of the conclusions drawn from the results, is instrumental for increasing public trust in science and reducing the spread of misinformation.
\item Academia should reward the creation, and long-term maintenance of software tools as much as they value new publications and awarded grants to combat the short lifespan of bioinformatics tools.
\item When we increase diversity in academia, we all win (10.15252/embr.202051994).
\item Humans are allergic to change. They love to say, ``We've always done it this way.'', I try to fight that. That's why I have a clock on my wall that runs counter-clockwise. - Grace Hopper
\end{enumerate}
